\section{Sąsiedztwo}

W badaniach do określenia sąsiedztwa użyto algorytmu
2-OPT. Zasada jego działania wydaje się najbardziej intuicyjna i oczywista.
W prezentowanej trasie jeden wierzchołek zostaje zamieniany
z innym wierzchołkiem i dzięki temu otrzywyane jest nowe rozwiązanie, które
jest dalej analizowane. Dzięki tak prostej konstrukcji reprezentacji
sąsiedztwa bardzo łatwo określić rozmiar sąsiedztwa. Wielkość ta jest
równa uzależniona od $n$, które oznacza rozmiar instancji, a więc ilość
wierzchołków w grafie. Wielkość sąsiedztwa można określić jako:

$$ C_{n}^{2} = {n\choose 2} = \frac{n!}{2(n-2)!} = \frac{n \cdot (n-1)}{2}$$