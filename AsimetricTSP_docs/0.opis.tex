\section{Opis}

Celem w asymetrycznym problemie komiwojażera jest odnalezienie
minimalnego cyklu Hamiltona w grafie pełnym skierowanym.
Asymetria w takim grafie związana jest z różnymi wagami
umieszczonymi na poszczególnych łukach. Tak więc łuk wychodzący z wierzchołka
$u$ do wierzchołka $v$ może mieć inną wagę niż łuk
od wierchołka $v$ do wierchołka $u$. Problem komiwojażera w rozpatrywanej
formie jest problemem optymalizacyjnym należącym do klasy problemów
\emph{NP-trudnych}.

Zadanie, którego dotyczy niniejszy raport polegało na utworzeniu czterech
algorytmów, które w sposób niedokładny będą rozwiązywać wspomniany
wcześniej problem. Takie rozwiązania mają być alternatywą dla żmudnych
i bardzo długich obliczeń związanych ze znajdowaniem rozwiązania
dokładnego w ogromnej przestrzeni rozwiązań.
W rozwiazaniu opisywanego zadania użyto następujących algorytmów.

\begin{enumerate}

\item Algorytm losowy --- polega na losowym wyborze kolejności
odwiedzanych wierzhołków i wyborze punktu dla, którego funkcja
oceny jest osiąga wartość minimalną.

\item Algorytm Greedy --- punktem startowym dla tego algorytmu jest
rozwiązanie losowe. W kolejnych iteracjach przebiegu algorytmu przeszukiwane
jest sąsiedztwo celem znalezienia lepszego rozwiązania. Znalezienie takiego
rozwiązanie związane jest z przerwaniem dalszego przeszukiwania sąsiedztwa
oraz przeszukiwaniem nowopowstałego rozwiązania.
Algorytm przerywa działanie, jeżeli nie znajduje lepszego rozwiązania w
całym obszarze sąsiedztwa.

\item Algorytm Steepest --- punktem startowym dla tego algorytmu jest
rozwiązanie losowe. W kolejnych iteracjach przebiegu algorytmu przeszukiwane
jest sąsiedztwo celem znalezienia lepszego rozwiązania. Znalezienie takiego
rozwiązanie nie związane jest z przerwaniem dalszego przeszukiwania sąsiedztwa
jak to miało miejsce w przypadku rozwiązania Greedy. Przeszukiwane jest
całe sąsiedztwo i spośród niego wybierane rozwiązanie o minimalnej wartości
funkcji oceny.
Algorytm przerywa działanie, jeżeli nie znajduje lepszego rozwiązania w
całym obszarze sąsiedztwa.

\item Własny algorytm --- algorytm wybiera losowy punkt spośród wszystkich
wierzchołków grafu skierowanego, i od tego punktu rozpoczyna się
przetwarzanie. Do listy odwiedzanych wierzchołków dodawane są kolejno
punkty, które są najbliższe aktualnemu i jeszcze nie zostały odwiedzone.

\item Symulowane Wyżarzanie (ang. Simulated Annealing) --- algorytm oparty 
na idei wyżarzania w metalurgii.
Na początku wybierane jest losowe rozwiązanie. Każde kolejne jest wyznaczane 
losowo ze zbioru rozwiązań sąsiadujących. Przechodzimy do nowego rozwiązania,
jeżeli rozwiązanie to poprawia najlepsze znane rozwiązanie, a jeżeli nie to 
sprawdzamy prawdopodobieństwo przejścia do tego nowego rozwiązania. 
Prawdopodobieństwo to jest obliczane
na podstawie obecnej temperatury, która spada z każdą pętlą algorytmu, oraz
odległości pomiędzy rozwiązaniem obecnym a kandydatem do przejścia. Im niższa
temperatura tym prawdopodobieństwa przejścia jest większe, więc częściej 
algorytm przechodzi do rozwiązań nie poprawiających wyniku, co może spowodować
wyskoczenie z optimum lokalnego.

\item Tablica Tabu (ang. Tabu Search) --- jest to algorytm, którego przejście do
następnego rozwiązania ograniczone jest przez tablicę tabu, która zawiera daną
ilość wykonanych przejść i zakazuje ich powtórzenia. Na początku wybierane jest
rozwiązanie losowe. Przy przejściu do następnych rozwiązań zawsze wybierany jest
nalepszy z sąsiadów.
Przejście jest możliwe, jeżeli nie zawiera się ono w tablicy tabu, lub jest w 
zbiorze aspiracji (ang. Aspiration Set), czyli zbiorze rozwiązań, 
które poprawiają nalepsze znane do tej pory rozwiązanie.

\end{enumerate}

