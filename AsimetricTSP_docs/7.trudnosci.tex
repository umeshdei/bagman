\section{Napotkane trudności}

\subsection{Błędy implementacji}

W czasie implementacji rozwiązania nie obyło się bez problemów. Bardzo
dużo kłopotów przyspożyły trudne do zlokalizowania bugi. W lokalizacji
pomogło narzędzie gdb, dzieki któremu można było dokonać krok po kroku
fragmenty programu, co do których było podejrzenie że działają niepoprawnie.
Dodatkowo, pomocna okazała się także analiza napisanego wcześniej kodu,
a także inspekcje kodu przeprowadzone w sposób kontroli krzyżowej
przez twórców rozwiązania.

\subsection{Format plików TSP}

Spore trudności podczas pracy były również związane z obsługą plików 
zawierających przykładowe instancjie problemu TSP, pobrane ze strony
internetowej \url{http://comopt.ifi.uni-heidelberg.de/software/TSPLIB95/}.
Nie można było znaleźć biblioteki umożliwiającej sprawne przetwarzanie
plików wejściowych pobranych ze wskazanego źródła. Konieczna okazała
się implementacja własnego modułu pozwalającego zaczytać dane.

\subsection {Czas przetwarania}

Wykonanie eksperymentów wiązało się z wielogodzinnymi obliczeniami. W
celu wykonania miarodajnych symulacji, konieczne było wielokrotne
ponawianie przetwarzania dla tych samych instancji, by na podstawie
tego można było wyliczyć średnie rozwiązanie oraz odchylenie
standardowe. Nie da się ukryć, że dla dużych instancji problemów
czas przetwarzania nie należał do krótkich, a w związku z tym
czas uzykania wyników końcowych eksperymentu był bardzo duży. Nie można
sobie było pozwolić na kilkukrotne ponawianie eksperymentów.
