\section{Porówannie wykorzystanych algorytmów}

Badania zostały przeprowadzone na komputerze Politechniki Poznańskiej przygotowanym do 
wykonywania dużych obliczeń (HPC --- High Performance Compuning). Dane generowane były 
poprzez losowe rozmieszczenie punktów (od 50 do 400), a następnie obliczenie odległości 
między nimi. Tak przygotowane dane były wczytywane przez program i na nich wykonywane były 
obliczenia. Algorytmy Greedy i Steepest uruchamiane były 10 razy, za każdym razem i innym 
runktem początkowym. Algorytm Random uruchamiany był 10 razy, za każdym razem z innym 
limitem powtórzeń (od 20 000 do 200 000). Ostatnim alogrytmem była prosta Heurystyna naszego
autorstwa, również uruchamiana 10 razy, za każdym razem z innym miastem początkowym.

Wyżej wymienione obliczenia były przeprowadzana 10 razy dla każdej z instancji, aby dzięki temu 
uniknąc przekłamania wyników uzyskanych poprzez specyficzny układ danych wejściowych.

\subsection{Odległość od optimum}

\subsection{Czas działania}

\subsection{Jakość/czas}

\subsection{Średnia ilośc kroków algorytmów}