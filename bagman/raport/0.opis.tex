\section{Opis}

Symetryczny problem komiwojażera polega na znalezieniu minimalnego cyklu Hamiltona
w nieskierowanym grafie pełnym. Należy on do klasy problemów NP-trudnych. 

Danymi wejściowymi jest $ n $ miast. Z każdego z miasta można przejechać do dowolnego
innego miasta nie odwiedzając innych miast po drodze.

Rozwiązaniem problemu komiwojażera jest permutacja miast. Funkcję celu można przedstawić
następująco, jako funkcję:

$$ (\min) \sum_{i=1}^{i < n} dist(i, (i+1) \% n ) $$

W poniższym projekcie użyto czterech algorytmów metaheurystycznych rozwiązujących 
dany problem. 

\begin{enumerate}

\item Algorytm losowy (R) --- polega na losowym wybieraniu kolejności odwiedzanych punktów i 
porównywaniu otrzymanego wyniku z dotychczasowo najlepszym.

\item Algorytm Greedy (G) --- algorytm ten startuje z losowego ustawienia. Przeszukuje 
sąsiedztwo w celu znalezienia lepszego rozwiązania. Po znalezieniu takiego, zaprzestaje 
poszukiwań i przechodzi do lepszego rozwiązania rozpoczynając od nowa przeglądanie sąsiedztwa.
Zaprzestaje swoje obliczenia, gdy nie znajduje lepszego rozwiązania.

\item Algorytm Steepest (S) --- algorytm ten startuje z losowego ustawienia. Przeszukuje 
sąsiedztwo celem znalezienia lepszego rozwiązania. Po przejrzeniu całego sąsiedztwa, przechodzi 
do najlepszego znalezionego rozwiązania rozpoczynając od nowa przeglądanie sąsiedztwa.
Zaprzestaje swoje obliczenia, gdy nie znajduje lepszego rozwiązania.

\item Prostą heurystyka(H) --- algorytm ten wybiera losowy punkt, od którego rozpoczyna obliczenia.
Następnie dodaje kolejne odwiedzane punkty, jako kryterium stosując odległość punktu od punktu, w 
którym obecnie się znajduje. Jest to algorytm Nearest Neighbour.

\item Symulowane wyżarzanie(SA) --- algorytm ten symuluje zachowanie wyżarzania występującego w 
przyrodzie. Rozpoczyna się z temperaturą, która sukcesywnie się obniża. Im wyższa temperatura, tym
możliwe większe zmiany uszeregowania (możliwe jest opuszczenie minimum lokalnego). Wraz ze spadkiem 
temperatury zakres przeszukiwanych rozwiązań jest coraz mniejszy i pozwala dokładniej zlokalizować
minimum.

\item Tabu search(TS) --- algorytm ten przegląda określoną część sąsiedztwa. Przy czym każdy ruch 
zapisuje na liście tabu. Jeśli nowy ruch występuje na liście tabu, ale poprawia wynik ogólny jest
on wykonywany. W przeciwnym razie algorytm przechodzi do następnego kroku. Jeśli natomiast krok 
ten nie występuje na liście, algorytm go wykonuje pomimo możliwości pogorszenia wyniku.


\end{enumerate}

