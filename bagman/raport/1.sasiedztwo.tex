\section{Sąsiedztwo}

W badaniach użyto algorytmu sąsiedztwa 2-OPT. W prezentowanej trasie 
2 wierzchołki zostają zamieniane miejscami i dzięki temu uzyskiwana jest 
nowa trasa.

Wielkość sąsiedztwa można ocenić po ilości dostępnych miast w trasie.
Inną trasę w naszym sąsiedztwie uzyskuje się poprzez zamianę 2 wierzchołków.
Jest to więc problem kombinatoryczny wyboru 2 wierzchołków z n. Rozmiar można 
opisać wzorem:

$$ C_{n}^{2} = {n\choose 2} = \frac{n!}{2(n-2)!} = \frac{n * (n-1)}{2}$$