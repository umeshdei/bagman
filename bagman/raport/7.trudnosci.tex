\section{Napotkane trudności}

\subsection{Format plików TSP}

Jedną z większych trudności napotkanych podczas pracy nad programem 
była obsługa plików z rozszerzeniem .tsp. Wymagane one były, aby sprawdzić
jak dokładne wyniki znajdują zaproponowane przez autorów algorytmy. Niestety
nie istnieje biblioteka, która umożliwiła by łatwe wczytanie tychże plików oraz dostępu
do danych znajdujących się w nich. Ponadto, dane zawarte w plikach .tsp były zapisane 
w kilku odmiennych formatach. Autorzy byli zmuszeni napisać własny
parser, który niestety nie obsługiwał wszystkich plików. 

\subsection{Czas działania aplikacji}

Celem porównania jak największej ilości różnych przypadków algorytmy były testowane dla 
różnych parametrów wejściowych. Dodatkowo uruchamiane były kilku krotnie aby zminimalizować
wpływ wyboru punktu początkowego. Sprawiało to, że ilość wykonań jednego algorytmu wynosiła 
kilkaset. Ponieważ niektóre przypadki potrafiły się liczyć do kilku godzin, sprawiało to, że
ogólny czas obliczeń wyniósł ponad kilka dni. Ze względu na brak czasu nie wszystkie wyniki 
udało się uzyskać.