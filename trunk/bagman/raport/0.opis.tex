\section{Opis}

Problem komiwojażera polega na znalezieniu minimalnego cyklu Hamiltona
w grafie pełnym. Należy on do klasy problemów NP-trudnych. 

W poniższym projekcie użyto czterech algorytów metaheurystycznych rozwiązująych 
dany problem. 

\begin{enumerate}

\item Algorytm losowy (R) --- polega na losowym wybieraniu kolejności odiwedzanych punktów i 
porównywaniu otrzymanego wyniku z dotychczasowo najlepszym.

\item Algorytm Greedy (G) --- algorytm ten startując z losowego ustawienia przeszukuje 
sąsiedztwo celem znalezienia lepszego rozwiązania. Po znalezieniu takiego zaprzestaje 
poszukiwań i przechodzi do lepszego rozwiązania rozpoczynając od nowa przeglądanie sąsiedztwa.
Zaprzestaje swoje obliczenia, gdy nie znajduje lepszego rozwiązania.

\item Algorytm Steepest (S) --- algorytm ten startując z losowego ustawienia przeszukuje 
sąsiedztwo celem znalezienia lepszego rozwiązania. Po przejrzeniu całego sąsiedztwa przechodzi 
do najlepszego znalezionego rozwiązania rozpoczynając od nowa przeglądanie sąsiedztwa.
Zaprzestaje swoje obliczenia, gdy nie znajduje lepszego rozwiązania.

\item Algorytm autorski (H) --- algorytm ten wybiera losowy punkt, od którego rozpoczyna obliczenia.
Następnie dodaje kolejne odwiedzane punkty, jako kryterium stosując odległość punktu od punktu, w 
którym obecnie się znajduje. Jest to przykład programowania dynamicznego.

\end{enumerate}

