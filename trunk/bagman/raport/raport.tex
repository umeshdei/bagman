\documentclass{article} 
\usepackage{polski} %moze wymagac dokonfigurowania latexa, ale jest lepszy ni� standardowy babel'owy [polish] 
\usepackage[utf8]{inputenc}
\usepackage[OT4]{fontenc} 
\usepackage{graphicx,color} %include pdf's (and png's for raster graphics... avoid raster graphics!) 
\usepackage{url} 
\usepackage{multirow}
% allows for temporary adjustment of side margins
\usepackage{chngpage}

% provides filler text
\usepackage{lipsum}

% just makes the table prettier (see \toprule, \bottomrule, etc. commands below)
\usepackage{booktabs}

\input{ustawienia.tex}

%\title{Sprawozdanie z laboratorium:\\Metaheurystyki i Obliczenia Inspirowane Biologicznie}
%\author{}
%\date{}


\begin{document}

\thispagestyle{empty} %bez numeru strony

\begin{center}
{\large{Sprawozdanie z laboratorium:\\
Metaheurystyki i Obliczenia Inspirowane Biologicznie}}

\vspace{3ex}

Część I: Algorytmy optymalizacji lokalnej, problem ATSP
%Cz��� II: Algorytmy optymalizacji lokalnej i globalnej, problem QAP
%Cz��� III: Eksperyment: ... (prezentacj� mo�na zrobi� w LaTeX/klasa beamer)

\vspace{3ex}
{\footnotesize\today}

\end{center}


\vspace{10ex}

Prowadzący: dr inż. Maciej Komosiński

\vspace{5ex}

Autorzy:
\begin{tabular}{lllr}
\textbf{Konrad Olczak} & inf80061 & SKiSR & kolczak87@gmail.com \\
\textbf{Paweł Róg} & inf80138 & SKiSR & pawelrog88@gmail.com \\
\end{tabular}

\vspace{5ex}

Zajęcia poniedziałkowe, 15:00.

\newpage




\section{Opis}

Problem komiwojażera polega na znalezieniu minimalnego cyklu Hamiltona
w grafie pełnym. Należy on do klasy problemów NP-trudnych. 

W poniższym projekcie użyto czterech algorytów metaheurystycznych rozwiązująych 
dany problem. 

\begin{enumerate}

\item Algorytm losowy (R) --- polega na losowym wybieraniu kolejności odiwedzanych punktów i 
porównywaniu otrzymanego wyniku z dotychczasowo najlepszym.

\item Algorytm Greedy (G) --- algorytm ten startując z losowego ustawienia przeszukuje 
sąsiedztwo celem znalezienia lepszego rozwiązania. Po znalezieniu takiego zaprzestaje 
poszukiwań i przechodzi do lepszego rozwiązania rozpoczynając od nowa przeglądanie sąsiedztwa.
Zaprzestaje swoje obliczenia, gdy nie znajduje lepszego rozwiązania.

\item Algorytm Steepest (S) --- algorytm ten startując z losowego ustawienia przeszukuje 
sąsiedztwo celem znalezienia lepszego rozwiązania. Po przejrzeniu całego sąsiedztwa przechodzi 
do najlepszego znalezionego rozwiązania rozpoczynając od nowa przeglądanie sąsiedztwa.
Zaprzestaje swoje obliczenia, gdy nie znajduje lepszego rozwiązania.

\item Algorytm autorski (H) --- algorytm ten wybiera losowy punkt, od którego rozpoczyna obliczenia.
Następnie dodaje kolejne odwiedzane punkty, jako kryterium stosując odległość punktu od punktu, w 
którym obecnie się znajduje.

\end{enumerate}



\section{Sąsiedztwo}

W badaniach użyto algorytmu sąsiedztwa 2-OPT. W prezentowanej trasie 
2 wierzchołki zostają zamieniane miejscami i dzięki temu uzyskiwana jest 
nowa trasa.

Wielkość sąsiedztwa można ocenić po ilości dostępnych miast w trasie.
Inną trasę w naszym sąsiedztwie uzyskuje się poprzez zamianę 2 wierzchołków.
Jest to więc problem kombinatoryczny wyboru 2 wierzchołków z n. Rozmiar można 
opisać wzorem:

$$ C_{n}^{2} = {n\choose 2} = \frac{n!}{2(n-2)!} = \frac{n * (n-1)}{2}$$

\section{Porówannie wykorzystanych algorytmów}

Badania zostały przeprowadzone na komputerze Politechniki Poznańskiej przygotowanym do 
wykonywania dużych obliczeń (HPC --- High Performance Compuning). Dane generowane były 
poprzez losowe rozmieszczenie punktów (od 50 do 400), a następnie obliczenie odległości 
między nimi. Tak przygotowane dane były wczytywane przez program i na nich wykonywane były 
obliczenia. Algorytmy Greedy i Steepest uruchamiane były 10 razy, za każdym razem i innym 
runktem początkowym. Algorytm Random uruchamiany był 10 razy, za każdym razem z innym 
limitem powtórzeń (od 20 000 do 200 000). Ostatnim alogrytmem była prosta Heurystyna naszego
autorstwa, również uruchamiana 10 razy, za każdym razem z innym miastem początkowym.

Wyżej wymienione obliczenia były przeprowadzana 10 razy dla każdej z instancji, aby dzięki temu 
uniknąc przekłamania wyników uzyskanych poprzez specyficzny układ danych wejściowych.

\subsection{Odległość od optimum}

\subsection{Czas działania}

\subsection{Jakość/czas}

\subsection{Średnia ilośc kroków algorytmów}

\section{Wnioski}

% Złożność algorytmów
\subsection{Złożoność obliczeniowa prostego algorytmu heurystycznego}

Zgodnie z załączonymi wynikami można zauważyć bardzo którki czas działania prostej
heurystki zaproponowanej przez autorów. Jest to spowodowane liniową złożonością 
czasową $ O(n) $. Otrzymujemy dzięki temu bardzo sybki algorytm. Jego wadą jest 
możliwość otrzymania wyniku bardzo oddalonego od optimum. Dzieje się tak dlatego, 
że przy końcowych obliczeniach nie mamy już dużego wyboru następnych odwiedzanych 
punktów, więc ostatnie ścieżki do wyboru mogę być bardzo długie. Wynik działania 
tego algorytmu mógłby być podstawą do uruchomienia algorymtów Steepest lub Greedy, 
które sprawdziłyby, czy w sąsiedztwie nie leżą lepsze rozwiązania. Znacznie skróciło
by to ich czas działania, poprzez start w miejscu, które jest blisko optimium.
Jest jednak możliwość, że punkt początkowy leżałby w pobliżu jakiegoś optimum 
lokalnego, i zawsze zwracany były ten sam wynik, uniemożliwiając odkrycie innych 
optimów i dzięki temu polepszenie wyniku.

\subsection{Ilość iteracji}


\section{Napotkane trudności}

\subsection{Format plików TSP}

Jedną z większych trudności napotkanych podczas pracy nad programem 
była obsługa plików z rozszerzeniem .tsp. Wymagane one były, aby sprawdzić
jak dokładne wyniki znajdują zaproponowane przez autorów algorytmy. Niestety
nie istnieje biblioteka, która umożliwiła by łatwe wczytanie tychże plików oraz dostępu
do danych znajdujących się w nich. Ponadto, dane zawarte w plikach .tsp były zapisane 
w kilku odmiennych formatach. Autorzy byli zmuszeni napisać własny
parser, który niestety nie obsługiwał wszystkich plików. 

\subsection{Czas działania aplikacji}

Celem porównania jak największej ilości różnych przypadków algorytmy były testowane dla 
różnych parametrów wejściowych. Dodatkowo uruchamiane były kilku krotnie aby zminimalizować
wpływ wyboru punktu początkowego. Sprawiało to, że ilość wykonań jednego algorytmu wynosiła 
kilkaset. Ponieważ niektóre przypadki potrafiły się liczyć do kilku godzin, sprawiało to, że
ogólny czas obliczeń wyniósł ponad kilka dni. Ze względu na brak czasu nie wszystkie wyniki 
udało się uzyskać.

\section{Wstęp}

% \begin{figure}
% \begin{center}
% \includegraphics[width=0.4\textwidth]{rys_graf.pdf}
% \end{center}
% \caption{Przyk�adowy schemat z graphviz'a. Przerywane strza�ki oznaczaj�, �e wsz�dzie gdzie si� da u�ywamy grafiki wektorowej -- unikamy wstawiania bitmap do dokumentu. W niekt�rych przypadkach u�ycie bitmap jest uzasadnione (w celu szybkiego podgl�du na ekranie lub dla niezwykle skomplikowanych grafik, zawieraj�cych np.~setki tysi�cy obiekt�w).}
% \label{fig-schemat}
% \end{figure}


To jest przykladowy tekst w LaTeX -- pokazuje jak
% \begin{tightlist}
% \item wstawi� schemat stworzony graphviz'em (Rys.~\ref{fig-schemat}),
% \item wstawi� wykres stworzony gnuplotem (Rys.~\ref{fig-1Tdelta} i \ref{fig-3d}),
% \item zacytowa� literatur� sformatowan� przez bibtex~\cite{MiOIB,Goldberg-2002},
% \item odwo�ywa� si� do rysunk�w, cytowa� i cz��ci sprawozdania (np.\ rozdzia�~\ref{sec-eksperymenty}).
% \end{tightlist}


% \begin{figure}
% \begin{center}
% \includegraphics[width=0.8\textwidth]{rys_wykres2d.pdf}
% \end{center}
% \caption{Przyk�adowy wykres z gnuplota, terminal postscript, zamieniony na pdf za pomoc� programu epstopdf z dystrybucji LaTeX'a (czasem eps2pdf). R��nice $\Delta_{dir}$ warto�ci $p_{dir}$ dla k�ta $90^\circ$.}
% \label{fig-1Tdelta}
% \end{figure}

% \begin{figure}
% \begin{center}
% \includegraphics[width=0.9\textwidth]{rys_wykres3d.pdf}
% \end{center}
% \caption{Jeszcze jeden przyk�adowy wykres z gnuplota.}
% \label{fig-3d}
% \end{figure}




\section{Eksperymenty}
\label{sec-eksperymenty}

W tym rozdziale nic nie ma.




%%%%%%%%%%%%%%%% literatura %%%%%%%%%%%%%%%%

% \bibliography{mioib}
% \bibliographystyle{plain}


\end{document}

