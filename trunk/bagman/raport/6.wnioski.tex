\section{Wnioski}

% Złożność algorytmów
\subsection{Złożoność obliczeniowa prostego algorytmu heurystycznego}

Zgodnie z załączonymi wynikami można zauważyć bardzo którki czas działania prostej
heurystki zaproponowanej przez autorów. Jest to spowodowane liniową złożonością 
czasową $ O(n) $. Otrzymujemy dzięki temu bardzo sybki algorytm. Jego wadą jest 
możliwość otrzymania wyniku bardzo oddalonego od optimum. Dzieje się tak dlatego, 
że przy końcowych obliczeniach nie mamy już dużego wyboru następnych odwiedzanych 
punktów, więc ostatnie ścieżki do wyboru mogę być bardzo długie. Wynik działania 
tego algorytmu mógłby być podstawą do uruchomienia algorymtów Steepest lub Greedy, 
które sprawdziłyby, czy w sąsiedztwie nie leżą lepsze rozwiązania. Znacznie skróciło
by to ich czas działania, poprzez start w miejscu, które jest blisko optimium.
Jest jednak możliwość, że punkt początkowy leżałby w pobliżu jakiegoś optimum 
lokalnego, i zawsze zwracany były ten sam wynik, uniemożliwiając odkrycie innych 
optimów i dzięki temu polepszenie wyniku.

\subsection{Ilość iteracji}
