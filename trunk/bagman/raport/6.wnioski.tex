\section{Wnioski}

% Złożność algorytmów
\subsection{Złożoność obliczeniowa prostego algorytmu heurystycznego}

Zgodnie z załączonymi wynikami można zauważyć bardzo którki czas działania prostej
heurystki zaproponowanej przez autorów. Jest to spowodowane kwadratową złożonością 
czasową $ O(n^{2}) $, ponieważ musimy $ n $ razy przejrzeć $ n $ punktów, aby znaleźć
nalbliższy. Otrzymujemy dzięki temu bardzo sybki algorytm. Jego wadą jest 
możliwość otrzymania wyniku bardzo oddalonego od optimum. Dzieje się tak dlatego, 
że przy końcowych obliczeniach nie mamy już dużego wyboru następnych odwiedzanych 
punktów, więc ostatnie ścieżki do wyboru mogę być bardzo długie. Wynik działania 
tego algorytmu mógłby być podstawą do uruchomienia algorymtów Steepest lub Greedy, 
które sprawdziłyby, czy w sąsiedztwie nie leżą lepsze rozwiązania. Znacznie skróciło
by to ich czas działania, poprzez start w miejscu, które jest blisko optimium.
Jest jednak możliwość, że punkt początkowy leżałby w pobliżu jakiegoś optimum 
lokalnego, i zawsze zwracany były ten sam wynik, uniemożliwiając odkrycie innych 
optimów i dzięki temu polepszenie wyniku.

\subsection{Odległość od optimum}

Przedstawione wyniki prezentują, że odległość od optimum dla wykonanych obliczeń 
są bardzo zróżnicowane. Najbardziej stalinie prezentuje się algorytm autorski 
dając średnio 20\% nadłożenie trasy. 

Algorytmy Steepest i Greedy lepiej działają dla mniejszych instancji, przy dużych 
znalezione rozwiązanie jest drukrotnie gorsze od znanego optimum. Dla małych instancji
odległość wacha się na poziomie 150\%, by w największych osiągnąć prawie 300\%.

Najbardziej zmienny jest algorytm losowy. Uzyskane dane są niedeterministyczne. Średnia
odległość od optimum wynosi ponad 500\%, ale zdażają się przypadki, w których algorytm ten 
zwracał najlepsze wyniki (ok. 108\%). Ale zdażały sie również przypadki o odległości ponad
900\%.

\subsection{Uzyskane wyniki}
                                                                       
Zgodnie z uzyskanymi obliczeniami najlepszy algorytmem okazał się prosty algorytm 
heurystyczny. Uyzkuje on, w niektórych przypadkach, wyniki lepsze o ponad 50\% od
drugiego w zestawieniu algorytmu Greedy. Kolejnym algorytmem jest Steepest z wynikami 
nieznacznie gorszymi od algorytmu Greedy. Ostatni w zestawieniu znalazł się algorytm 
losowy.

\subsubsection{Prosty algorytm heurystyczny}

Sukces algorytmu autorskiego opiera się w dużej części na podejściu do problemu. 
Programowanie dynamiczne wybiera w daej chwili najbardziej optymalne rozwiązanie, 
dzięki czemu osiąga w ogólności bardzo dobre wyniki. Istnieją jednak sytuacje, gdy 
rozwiązania zaproponowane przez prosty algorytm heurystyczny może być ponad 2 razy 
gorsze od pozostałych algorytmów. Dzieje się tak w przypadku, gdy ostatnia wybrana 
trasa jest bardzo długa. Dlatego po otrzymaniu rozwiązania należałoby przejrzeć 
sąsiedztwo w poszukiawaniu rozwiązania lepszego umożliwiając, w niektórych przypadkach
usinięci najdłużeszej trasy.

\subsubsection{Algorytm losowy}

Słaby wynik algorytmu wiąże się z jego niedeterminizmem, jak i brakiem wykorzystania 
już osiągniętych informacji. Jest to jedyny algorytm, dla którego nie można określić 
złożoności, gdyż będzie on działał tak długo, aż jego praca zostanie przerwana. Losując 
rozwiązania na bardzo dużych danych wejściowych ciężko osiąnąć jest wynik nawet zbliżony 
do optimum. 

W przypadku małych instancji algorytm ten zwracał wyniki porównywalne z 
pozostałymi algorytmami w o wiele krótszym czasie (nie licząc czasu działania algorytmu
autorskiego). Dla $ n = 10 $ ilość permutacji wynosi:

$$ n! = 10! = 3~628~800 $$

co przy przykładowych 200~000 powtórzeń daje szansę w trafienie w rozwiązanie optymalne:

$$ \frac{200~000}{10!} * 100\% = 5,51\%  $$

Natomiast jeśli podniesiemy $ n = 50 $ otrzymujemy wyniki:

$$ n! = 50! = 3,04 * 10^{64} $$
$$ \frac{200~000}{50!} * 100\% = 6,58 * 10^{-58}\%  $$

A przy $ n = 400 $, czyli górnej granicy obliczeń:

$$ n! = 400! = 6,40 * 10^{870} $$
$$ \frac{200~000}{400!} * 100\% = 3,13 * 10^{-864}\%  $$

Jak więc widać dla dużych $ n $ szanse trafienia w rozwiązanie optymalne są równe $ 0 $.

Algorytm ten nie wykorzystuje różwnież informacji o już znalezionych rozwiązaniach 
w przypadku generowania nowego, dzięki czemu wygenerowane rozwiązania nie przybliżają
go do optimum.

\subsubsection{Algorytm Greedy i Steepest}

Wyniki uzyskane przez algorytm Greedy są lepsze niż uzyskane przez algorytm Steepest. 
Jest to spowodowane przechodzeniem do najlepszego rozwiązania przez algorytm Steepest. 
Dzięki temu wybiera on najczęściej ścieżkę do najbliższego optimum lokalnego. Algorytm
Greedy natomiast wybiera pierwsze znalezione rozwiązanie, które jest lepsze od dotychczas 
znalezionego. dzięki temu nie zbliża się gwałtownie do żadnego z optimów lokalnych, lecz 
raczej przeszukuje obszar tak długo, aż znajdzie ślepy zaułek.


\subsection{Czas działania}

\it
W zestawieniu nie został ujęty prosty algorytm heurystyczny z powodu zbyt krótkich czasów 
działania. Dla instancji o rozmiarach ponad 700 elementów pokazywał on czas działania 0.00, 
podczas gdy alogrytm Steepest potrzebował podan 1,5h.
\rm

\subsubsection{Algorytm Greedy}

Czas działania tego algorytmu dla większych instancji jest lepszy o ok. 50 - 70 \% w porównaniu 
do algorytmu Steepest. Pomimo, że potrzebuje on więcej iteracji, przejście do następnego kroku 
zajmuje mniej czasu, gdyż następuje od razu po znalezieniu lepszego rozwiązania. Jego złożoność 
wynosi następująco:

\subsubsection{Algorytm Steepest}

Algorytm ten jest ok. 2 razy gorszy od algorytmu Greedy.Wynika to głównie z fakty, że przegląda on
wszystkie rozwiązania zanim podejmie decyzję o następnym kroku. 

\subsubsection{Algorytm losowy}

Algorytm ten działa niederministycznie, dlatego potrafi osiągnąć szybsze czasy działania dla 
większych instancji. 

\subsection{Ilość iteracji}

Porównując ilość iteracji algorytmów Greedy i Steepest można zauważyć, że algorytm 
Steepest wykonuje ok. 4 razy mniej kroków od algorytmu Greedy. Jest to spowodowane 
dokładnym przeglądaniem przestrzeni sąsiedztwa i wybierania najleprzego rozwiązania.

Nawiązując jednak do ilości przejrzajnych rozwiązań można zauważyć, że algorytm
Greedy przegląda ich ok. 50\% mniej, co istotnie wpływa na szybkość obliczeń.
