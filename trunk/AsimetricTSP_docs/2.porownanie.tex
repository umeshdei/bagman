\section{Porówannie wykorzystanych algorytmów}

\subsection{Parametry sprzętowe}

Badania zostały przeprowadzone na kilku węzłach spośród dostępnego na
Politechnice klastra obliczeniowego. Jednakże każdy pojedynczy pomiar
odbywał się przy obliczeniach wykonywanych na jednaj maszynie.
Każda z jednostek klastra posiada następujące parametry sprzętowe:
\begin{itemize}
 \item CPU : Intel(R) Xeon(R) CPU X3230@2.66GHz
 \item Pamięć : 4GB
\end{itemize}
Pomimo, że procesory dysponowały czterema rdzeniami, nie zdecydowano się
na zrównoleglenie procesu obliczeń oraz zrównolelenie samego procesu
wykonywania testów. W zamian za to, każdy typ algorytmu testowany był
na innej maszynie fizycznej.

\subsection{Dane wejściowe}

Wygenerowane na potrzeby testów dane charakterysowały się
losowym rozmieszczeniem punktów. Generator instancji ze względu na
asymetrię problemu musiał uwzględniać różne odległości między punktami
w zależności od kierunku łuku w grafie. Tak więc początkowo losowane były
pozycje punktów. Na podstawie pozycji wyznaczana była odległość
zgodnie z kierunkiem wierzchołków od $u$ do $v$. Odległość od $v$ do $u$
z kolei była wyznaczana na podstawie poprzednio wyliczonej wartości
jednakże była modyfikowana o pewną losową wartość $\pm rand()$.


Dane wejściowe posiadały w zależności od konfiguracji od 50 do 400
wierzchołków. Przygotowane dane były wczytywane przez program z wcześniej
utworzonych plików, a na podstawie położenia zaczytanych wierzchołków
oraz odległości między nimi wykonywane były 
obliczenia. Wszystkie testowane algorytmy uruchamiane
były po 10 razy dla każdej z testowanych instancji. Kolejne uruchomienia
różniły się rozwiązaniem początkowym, dlatego można się było spodziewać
różnych wyników dla każdego spośród wywołań. Algorytm 
losowy był uruchamiony dodatkowo dla każdej instancji po 10 razy ze zmiennym
limitem losowań w granicach 20~000 do 200~000 powtórzeń.

\subsection{Odległość od optimum}

\subsection{Czas działania}

\subsection{Jakość/czas}

\subsection{Średnia ilość kroków algorytmów}